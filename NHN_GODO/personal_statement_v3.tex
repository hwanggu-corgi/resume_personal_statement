\documentclass[12pt]{article}
\usepackage[margin=2.5cm]{geometry}
\usepackage{enumerate}
\usepackage{amsfonts}
\usepackage{amsmath}
\usepackage{fancyhdr}
\usepackage{amsmath}
\usepackage{amssymb}
\usepackage{amsthm}
\usepackage{mdframed}
\usepackage{graphicx}
\usepackage{subcaption}
\usepackage{adjustbox}
\usepackage{listings}
\usepackage{xcolor}
\usepackage{booktabs}
\usepackage[utf]{kotex}
\usepackage{hyperref}
\usepackage{accents}

\definecolor{codegreen}{rgb}{0,0.6,0}
\definecolor{codegray}{rgb}{0.5,0.5,0.5}
\definecolor{codepurple}{rgb}{0.58,0,0.82}
\definecolor{backcolour}{rgb}{0.95,0.95,0.92}

\lstdefinestyle{mystyle}{
    backgroundcolor=\color{backcolour},
    commentstyle=\color{codegreen},
    keywordstyle=\color{magenta},
    numberstyle=\tiny\color{codegray},
    stringstyle=\color{codepurple},
    basicstyle=\ttfamily\footnotesize,
    breakatwhitespace=false,
    breaklines=true,
    captionpos=b,
    keepspaces=true,
    numbers=left,
    numbersep=5pt,
    showspaces=false,
    showstringspaces=false,
    showtabs=false,
    tabsize=1
}

\lstset{style=mystyle}

\pagestyle{fancy}
\renewcommand{\headrulewidth}{0.4pt}
\rhead{Notes}

\begin{document}
\title{Personal Statement V3}
\maketitle

\textbf{Question: NHN godo의 SW개발 직무에 본인이 선발되어야 하는 이유에 대해 강점을 기반으로 기재해 주세요 (500자 이내)}

\bigskip

(요구사항에 맞추어 생각한 대로 구현할 수 있는 개발자)

\bigskip

저는 2년의 industry 경험과 요구사항에 맞추어 구현할 수 있는 개발자 입니다. (need to come back later)

\bigskip

저는 요구사항에 맞추어 생각한 대로 구현한 경험이 있고 그리고 그것을 더 잘 하고자 끝임없이 배우는 사람입니다.
저는 SiteMax Systems inc. 에서 웹 어플을 만들기전 COO와 CEO와 다른동료들에게 요구사항을 물어보고 이해하고 실행 가능한지 확인하고 만들었습니다.
그리고 지금 토론토 대학에서도 C 언어를 사용해 file system을 만들때 필요 문구사항을 읽고 모르면 물어보며 요구사항에 맞추어 프로그램을 만들었습니다

\bigskip

\bigskip

저는 지금 대학원을 진학하고자 필요한 3학년 코스 OS 프로그램을 듣고있습니다.
OS 프로그램을 구현하기위해 필요한 C 언어 지식을 \texttt{K.N King's C programming}에 있는 문재를 풀고 독학하여
언어를 배우고 지금 코스를 듣으며 공부하고 있습니다.

\bigskip

큰 앱을 만들때 가장 어려운 것중에 하나가 기존에 잘 동작된것들이 수정 후 동작이 잘 안되는것입니다.
저는 Team Treehouse에서 공부할떄 큰 앱을 만들며 수정을 할때 아니면 새로운것을 만들때 늘 기존에 잘 동작된것들이 동작이 잘 안됬었습니다 그리고 그 문제를 찾기도 어려웠습니다.
저는 그 문제를 해결하고자 TDD를 사용하였고 그 결과 큰 앱을 만드는데 보다 안정적이고 부담없이 만들 수 있었습니다.
그리고 저는 이 TDD 스킬을 지금 토론토 대학에서 C 언어와 shell scripting을 함께 이용해 FUSE기반 작은 OS를 만드는데 사용했습니다.

\bigskip

저는 2년 동안 회사에서 javascript 을 사용했고 python을 즐겨쓰는 개발자 입니다.

\bigskip

제게 기회를 주신다면 제 2년의 industry 경험과 프로그래밍 지식와 끝임없이 배우는 자세로 이 회사에 기여를 하겠습니다. 감사합니다.

\underline{\textbf{Notes}}

\begin{itemize}
    \item Key Point
    \begin{itemize}
        \item \#생각한대로 \#원하는대로 \# 개발 잘할수있지?
        \item Java, Python, Javascript, PHP 언어중 한개
        \begin{itemize}
            \item 저는 Python 하고 Javascript에 사용경력이 많은 사람입니다
            \item 저는 2008년도때 파이톤을 꾸준이 사용해왔습니다
            \item 저는 Javascript을 2018년 1월 부터 2019 12월 말까지 2년동안 회사 SiteMax Systems에서 Angular.js와 함께 사용했습니다
            \item
        \end{itemize}
        \item 상대방의 그리고 동료의 의견을 잘 이해하고 좋은 방향으로 이끌어 나갈 수 있는 능력
        \begin{itemize}
            \item 저는 동료들과 얘기하고 협업하며 앱을 만든 경험을 갖고 있습니다
            \item 저는 SiteMax System에서 동료와 코드리뷰할때 경청하고 작게 논의하여 기존 앱을 수정하고 완성품을 만들어왔습니다
        \end{itemize}
        \item 요구사항에 맞추어 생각한 대로 구현할 수 있는 정도
        \begin{itemize}
            \item 저는 siteMax에서도 지금도 요구사항에 맞추어 프로그램을 구현 하고 있습니다
            \item 지금 토론토 대학에서도 필요 문구사항을 읽고 요구사항에 맞추어 프로그램을 만들고 있습니다
        \end{itemize}

        \bigskip

        \underline{\textbf{Version 2}}

        \bigskip

        \begin{itemize}
            \item 저는 요구사항에 맞추어 생각한 대로 구현한 경험이 있고 그리고 그것을 더 잘 하고자 끝임없이 배우는 사람입니다
            \item 저는 siteMax 에서 웹 어플을 만들기전 COO와 CEO와 다른동료들에게 요구사항을 물어보고 이해하고 실행 가능한지 확인하고 만들었습니다
            \item 그리고 Team Treehouse에서 파이톤 기반 큰 어플을 만들때 TDD를 활용해 서버 요구사항을 충족해가며 만들었습니다
            \item 지금 토론토 대학에서도 필요 문구사항을 읽고 모르면 물어보며 요구사항에 맞추어 프로그램을 만들고 있습니다
            \item 제가 만약에
        \end{itemize}
        \item 새로운것에 부담감이 없아야 함
        \begin{itemize}
            \item 저는 새로운것에 배우는것을 즐거워 하는 사람입니다
            \item 토론토 대학원에 가고싶었고 3학년 코스를 들어야 했습니다.
            \item 그중에 하나가 OS 프로그래밍 과목을 들어야 했는데 C 언어 지식이 필요했습니다.
            \item 저는 C 언어를 제 여인과 함께 \texttt{K.N King's C programming} 문제를 풀어 공부를 하였고 그 결과 지금 저는 토론토 대학에서 OS을 공부하고 있습니다
        \end{itemize}
        \item 함께일해요***

        \item 기존에 잘 동작된것들이 동작이 잘 안된다

        \begin{itemize}
            \item 큰 앱을 만들때 가장 어려운 것중에 하나가 기존에 잘 동작된것들이 수정 후 동작이 잘 안되는것입니다.
            \item 저는 Team Treehouse에서 공부할떄 큰 앱을 만들며 수정을 할때 아니면 새로운것을 만들때 늘 기존에 잘 동작된것들이 동작이 잘 안됬었습니다 그리고 그 문제를 찾기도 어려웠습니다
            \item 저는 그 문제를 해결하고자 TDD를 사용하였고 그 결과 큰 앱을 만드는데 보다 안정적이고 부담없이 만들 수 있었습니다
            \item 그리고 저는 이 TDD 스킬을 지금 토론토 대학에서 C 언어와 shell scripting을 함께 이용해 FUSE기반 작은 OS를 만드는데 사용했습니다
        \end{itemize}

        \bigskip

        \underline{\textbf{Version 2}}

        \bigskip

        \begin{itemize}
            \item 저는 TDD를 즐겨쓰는 사람입니다
            \item 처음에 큰 앱을 만들며 수정을 할때 아니면 새로운것을 만들때 늘 기존에 잘 동작된것들이 동작이 잘 안됬었습니다 그리고 그 문제를 찾기도 어려웠습니다
            \item 저는 그 문제를 해결하고자 Team Treehouse에서 TDD를 습득하였고 그 결과
            큰 앱을 만드는데 보다 안정적이고 부담없이 만들 수 있었습니다
            \item 그리고 저는 이 TDD 스킬을 지금 토론토 대학에서 C 언어와 shell scripting을 함께 이용해 FUSE기반 작은 OS를 만드는데 사용했습니다
        \end{itemize}
    \end{itemize}
    \item 자기소개서 작성 요령에 대해서 \href{https://brunch.co.kr/@hklim/11}{link}
    \begin{itemize}
        \item Step 1: 소제목 작성
        \begin{itemize}
            \item Ex. 진정성을 기휙에 담는 홍보인
            \item Ex 2. 누구보다 바르게 배우는 에이스
        \end{itemize}
        \item Step 2: 두괄식으로 작성할 것
        \item Step 3: 자신의 경험을 STAR 기법으로 작성할것
        \item Step 4: 회사에 기여할 점을 반드시 작성할 것
    \end{itemize}
\end{itemize}




\end{document}