\documentclass[12pt]{article}
\usepackage[margin=2.5cm]{geometry}
\usepackage{enumerate}
\usepackage{amsfonts}
\usepackage{amsmath}
\usepackage{fancyhdr}
\usepackage{amsmath}
\usepackage{amssymb}
\usepackage{amsthm}
\usepackage{mdframed}
\usepackage{graphicx}
\usepackage{subcaption}
\usepackage{adjustbox}
\usepackage{listings}
\usepackage{xcolor}
\usepackage{booktabs}
\usepackage[utf]{kotex}
\usepackage{hyperref}
\usepackage{accents}

\definecolor{codegreen}{rgb}{0,0.6,0}
\definecolor{codegray}{rgb}{0.5,0.5,0.5}
\definecolor{codepurple}{rgb}{0.58,0,0.82}
\definecolor{backcolour}{rgb}{0.95,0.95,0.92}

\lstdefinestyle{mystyle}{
    backgroundcolor=\color{backcolour},
    commentstyle=\color{codegreen},
    keywordstyle=\color{magenta},
    numberstyle=\tiny\color{codegray},
    stringstyle=\color{codepurple},
    basicstyle=\ttfamily\footnotesize,
    breakatwhitespace=false,
    breaklines=true,
    captionpos=b,
    keepspaces=true,
    numbers=left,
    numbersep=5pt,
    showspaces=false,
    showstringspaces=false,
    showtabs=false,
    tabsize=1
}

\lstset{style=mystyle}

\pagestyle{fancy}
\renewcommand{\headrulewidth}{0.4pt}
\rhead{Notes}

\begin{document}
\title{Personal Statement V3}
\maketitle

\textbf{Question: NHN godo의 SW개발 직무에 본인이 선발되어야 하는 이유에 대해 강점을 기반으로 기재해 주세요 (500자 이내)}

\bigskip

(300개 넘는 작은 웹 앱과 모바일 앱을 함께 만든 개발자)

\bigskip

개발을 할때 만든 솔루션이 한번하면 되돌아 오지 않고 나중에 다른 맴버에게 피해 주지 않는게 함깨일해요의 핵심입니다.
저는 SiteMax Systems inc.에 개발자로 일하면서 초창기에 성과내야지 하는 마음에 한꺼번에 많이 하려고 했었습니다.
하지만 새로운 앱을 하나하나 만들면서 기존에 만들었던 앱들을 계속 수정 해야했고 일은 쌓아져가기만 했습니다.
이때 제 선배님의 조언을 들었습니다 "형모야. 일을 한번하면은 되돌아 오지 않게 않는게 핵심이란다" 였습니다.
그 이후 선배에게 실현 가능한지 물어보고 일을 하나하나씩 끝내고 선후배와 함께 코드리뷰를 했고 제 자신의 주장을 말하는 것보다 경청을 했고 생각이 다르면 존중히 논의하고 부족하면 보고 배웠습니다.
그 결과 전 제 동료하고 하나하나 끝내며 2년동안 함께 300개 넘는 앱을 만들 수 있었습니다.
제게 기회를 주시면 2 년의 경험과 한번하면 되돌아 오지 않는 자세로 팀과 함께 일하며 회사의 발전에 기여하겠습니다. 감사합니다.

\underline{\textbf{Notes}}

\begin{itemize}
    \item 사람들에게 칭찬받는 점이 뭐에요?
    \begin{itemize}
        \item 한번하면은 끝을 보는 끈기 입니다
        \item 부족하면 배우고 발전하려고 하는 자세입니다
        \item 꼼꼼함
    \end{itemize}
    \item 사람들한테 자랑하고 싶은 자신의 특징은 뭐에요?
    \item 그런 특징들과 관련된 경험은 뭐가 있어요?
    \item 그런 특징들이 하려는 일과 무슨 상관이 있나요?
    \item Experience
    \begin{itemize}
        \item 저는 초기에 성과내야지 하는 마음에 한꺼번에 많이 하려고 했던 개발자 였습니다.
        \item 저는 5 개의 함수나 앱 만들면 기존에 했던 5개의 수정을 더해 총 10개를 했어야 했습니다.
        \item 저는 초창기때 선배의 조언을 많이 들었습니다.
        \item 그중에서 가장 기억나는것이 "형모야. 일을 한번하면은 되돌아 오지 않게 않는게 핵심이란다" 입니다.
        \item 저는 그 이후 후배와 함꼐 일하고 코드리뷰 하면서 성과 생각을 내리고 일을 한번하면 되돌아 오지 않는 제품을 만드는데 주력했습니다.
        \item 동료하고 덕담을 주고받고 틀리면 받
        \item 저는 토론토 대학에서 OS 시스템을 공부하며 피해주지 않고
        \item 한번하면 되돌아 오지 않고 요구사항에 맞추고 solution이 다른 맴버에게 피해 주지 않는게 함깨일해요의 핵심입니다.
        \item 제게 기회를 주시면 피해주지 않는 개발자로서 함께 일하며 회사에 기여하겠습니다. 감사합니다.
    \end{itemize}
    \item Key Point
    \begin{itemize}
        \item \#생각한대로 \#원하는대로 \# 개발 잘할수있지?
        \item Java, Python, Javascript, PHP 언어중 한개
        \item 상대방의 그리고 동료의 의견을 잘 이해하고 좋은 방향으로 이끌어 나갈 수 있는 능력
        \item 요구사항에 맞추어 생각한 대로 구현할 수 있는 정도
        \item 새로운것에 부담감이 없아야 함
        \item 함께일해요***

        \item 기존에 잘 동작된것들이 동작이 잘 안된다

    \end{itemize}
    \item 자기소개서 작성 요령에 대해서 \href{https://brunch.co.kr/@hklim/11}{link}
    \begin{itemize}
        \item Step 1: 소제목 작성
        \begin{itemize}
            \item Ex. 진정성을 기휙에 담는 홍보인
            \item Ex 2. 누구보다 바르게 배우는 에이스
        \end{itemize}
        \item Step 2: 두괄식으로 작성할 것
        \item Step 3: 자신의 경험을 STAR 기법으로 작성할것
        \item Step 4: 회사에 기여할 점을 반드시 작성할 것
    \end{itemize}

    \bigskip
    \item 살아남는 자기소개서의 7가지 요소 \href{https://brunch.co.kr/@kuehyunpark/6}{link}
    \begin{itemize}
        \item 맞춤형 자기소개서
        \item 차별화된 자기소개서
        \item 출제의도에 맞는 자기소개서
        \item 설득력 있는 자기소개서
        \item 간결한 자기소개서
        \item 솔직한 자기소개서
    \end{itemize}

\end{itemize}




\end{document}