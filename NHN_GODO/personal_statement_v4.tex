\documentclass[12pt]{article}
\usepackage[margin=2.5cm]{geometry}
\usepackage{enumerate}
\usepackage{amsfonts}
\usepackage{amsmath}
\usepackage{fancyhdr}
\usepackage{amsmath}
\usepackage{amssymb}
\usepackage{amsthm}
\usepackage{mdframed}
\usepackage{graphicx}
\usepackage{subcaption}
\usepackage{adjustbox}
\usepackage{listings}
\usepackage{xcolor}
\usepackage{booktabs}
\usepackage[utf]{kotex}
\usepackage{hyperref}
\usepackage{accents}

\definecolor{codegreen}{rgb}{0,0.6,0}
\definecolor{codegray}{rgb}{0.5,0.5,0.5}
\definecolor{codepurple}{rgb}{0.58,0,0.82}
\definecolor{backcolour}{rgb}{0.95,0.95,0.92}

\lstdefinestyle{mystyle}{
    backgroundcolor=\color{backcolour},
    commentstyle=\color{codegreen},
    keywordstyle=\color{magenta},
    numberstyle=\tiny\color{codegray},
    stringstyle=\color{codepurple},
    basicstyle=\ttfamily\footnotesize,
    breakatwhitespace=false,
    breaklines=true,
    captionpos=b,
    keepspaces=true,
    numbers=left,
    numbersep=5pt,
    showspaces=false,
    showstringspaces=false,
    showtabs=false,
    tabsize=1
}

\lstset{style=mystyle}

\pagestyle{fancy}
\renewcommand{\headrulewidth}{0.4pt}
\rhead{Notes}

\begin{document}
\title{Personal Statement V4}
\maketitle

\textbf{Question: NHN godo의 SW개발 직무에 본인이 선발되어야 하는 이유에 대해 강점을 기반으로 기재해 주세요 (500자 이내)}

\bigskip

(요구사항에 맞추어 생각한 대로 구현할 수 있는 개발자)

\bigskip

Design 그리고 business requirement의 요구사항에만 맞춘 solution은 solution이 아닙니다.

\bigskip

저는 2년동안 회사에서 300개 넘는 작은 앱과 모바일 앱을 요구사항에 맞추어 구현한 경험이 있는 개발자 입니다.

\bigskip

저는 2018년 SiteMax Systems inc. 에서 Anguar.js 기반 웹 어플 하고 Angular 기반 모바일 앱을 만들었습니다.

\bigskip

바쁨에 빨리 해야한다는 마음에 task completion 에만 신경을 썼습니다. 그 결과 5 개를 끝내고 다른 5개를 할때 수정해야할 5개까지
더해 총 10개의 task를 했어야 했습니다. 선배의 꾸지람도 많이 듣었습니다.

\bigskip
무 에서 유를 창출하는데 어떻하면은 요구사항에 맞추어 잘 할 수 있을까? 가 제 가장 큰 고민 이었습니다.


\bigskip

저는 SiteMax System에서 일하면서 앱을 요구사항에 마추어 구현할때 또는 수정을 할때 많은 어려움을 겪었습니다.
그 중에 하나가 기존에 잘 동작된것들이 동작이 잘 안되고 수정하는 시간이 만드는 시간보다 길어지는 아픔이었습니다.
저는 이 부분을 해결하고자 저녁에 Team Treehouse에서 공부했고 Django 기반 Full Stack Development를
공부하며 큰 앱을 보다 안정적이게 수정하고 만드는것에 도움주는 Test Driven 개발법을 배웠습니다.

\bigskip

저는 지금 한층 더 나아가 토론토 대학에서 여름때 배운 C 언어와 함께 OS 시스템을 배우며 요구사항에 맞추어 프로그램을 구축하는 스킬을 hone 하고 있습니다.

\bigskip

저는 토론토 대학에서 C 언어 기반 파일 시스템을 만들때 필요한 아이디어와 이론을 proposal에 썼고 아이디어가 요구사항에 맞았을때 실행으로 옮겼습니다.
그리고 만들때는 C 언어와 shell script를 이용한 TDD를 사용하였고 그 결과 저는 파일 시스템을 흔들림 없이 구축 할 수 있었습니다.

\bigskip

제게 기회를 주신다면 제 2년의 industry 경험과 끝임없이 배우는 자세로 함께 잘 할 수 있지? 하면 함께 잘 할 수 있습니다
하는 개발자로서 이 회사에 기여를 하겠습니다. 감사합니다.

\underline{\textbf{Notes}}

\begin{itemize}
    \item Experience
    \begin{itemize}
        \item 저는 초기에 성과내야지 하는 마음에 한꺼번에 많이 하려고 했던 개발자 였습니다.
        \item 저는 5 개의 함수나 앱 만들면 기존에 했던 5개의 수정을 더해 총 10개를 했어야 했습니다.
        \item 저는 초창기때 선배의 조언을 많이 들었습니다.
        \item 그중에서 가장 기억나는것이 "형모야. 일을 한번하면은 되돌아 오지 않게 않는게 핵심이란다" 입니다.
        \item 저는 그 이후 후배와 함꼐 일하고 코드리뷰 하면서 성과 생각을 내리고 일을 한번하면 되돌아 오지 않는 제품을 만드는데 주력했습니다.
        \item 동료하고 덕담을 주고받고 틀리면 받
        \item 그리고 저는 선배의 말씀을 갖고 끝임없이 배우며 TDD를 배우고
        \item 한번하면 되돌아 오지 않고 요구사항에 맞추고 solution이 다른 맴버에게 피해 주지 않는게 함깨일해요의 핵심입니다.
        \item 제게 기회를 주시면 피해주지 않는 개발자로서 함께 일하며 회사에 기여하겠습니다. 감사합니다.
    \end{itemize}
    \item Key Point
    \begin{itemize}
        \item \#생각한대로 \#원하는대로 \# 개발 잘할수있지?
        \item Java, Python, Javascript, PHP 언어중 한개
        \item 상대방의 그리고 동료의 의견을 잘 이해하고 좋은 방향으로 이끌어 나갈 수 있는 능력
        \item 요구사항에 맞추어 생각한 대로 구현할 수 있는 정도
        \item 새로운것에 부담감이 없아야 함
        \item 함께일해요***

        \item 기존에 잘 동작된것들이 동작이 잘 안된다

    \end{itemize}
    \item 자기소개서 작성 요령에 대해서 \href{https://brunch.co.kr/@hklim/11}{link}
    \begin{itemize}
        \item Step 1: 소제목 작성
        \begin{itemize}
            \item Ex. 진정성을 기휙에 담는 홍보인
            \item Ex 2. 누구보다 바르게 배우는 에이스
        \end{itemize}
        \item Step 2: 두괄식으로 작성할 것
        \item Step 3: 자신의 경험을 STAR 기법으로 작성할것
        \item Step 4: 회사에 기여할 점을 반드시 작성할 것
    \end{itemize}

    \bigskip
    \item 살아남는 자기소개서의 7가지 요소 \href{https://brunch.co.kr/@kuehyunpark/6}{link}
    \begin{itemize}
        \item 맞춤형 자기소개서
        \item 차별화된 자기소개서
        \item 출제의도에 맞는 자기소개서
        \item 설득력 있는 자기소개서
        \item 간결한 자기소개서
        \item 솔직한 자기소개서
    \end{itemize}

\end{itemize}




\end{document}