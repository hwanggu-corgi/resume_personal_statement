\documentclass[12pt]{article}
\usepackage[margin=2.5cm]{geometry}
\usepackage{enumerate}
\usepackage{amsfonts}
\usepackage{amsmath}
\usepackage{fancyhdr}
\usepackage{amsmath}
\usepackage{amssymb}
\usepackage{amsthm}
\usepackage{mdframed}
\usepackage{graphicx}
\usepackage{subcaption}
\usepackage{adjustbox}
\usepackage{listings}
\usepackage{xcolor}
\usepackage{booktabs}
\usepackage[utf]{kotex}
\usepackage{hyperref}
\usepackage{accents}

\definecolor{codegreen}{rgb}{0,0.6,0}
\definecolor{codegray}{rgb}{0.5,0.5,0.5}
\definecolor{codepurple}{rgb}{0.58,0,0.82}
\definecolor{backcolour}{rgb}{0.95,0.95,0.92}

\lstdefinestyle{mystyle}{
    backgroundcolor=\color{backcolour},
    commentstyle=\color{codegreen},
    keywordstyle=\color{magenta},
    numberstyle=\tiny\color{codegray},
    stringstyle=\color{codepurple},
    basicstyle=\ttfamily\footnotesize,
    breakatwhitespace=false,
    breaklines=true,
    captionpos=b,
    keepspaces=true,
    numbers=left,
    numbersep=5pt,
    showspaces=false,
    showstringspaces=false,
    showtabs=false,
    tabsize=1
}

\lstset{style=mystyle}

\pagestyle{fancy}
\renewcommand{\headrulewidth}{0.4pt}
\rhead{Notes}

\begin{document}
\title{Personal Statement V2}
\maketitle

\textbf{Question: NHN godo의 SW개발 직무에 본인이 선발되어야 하는 이유에 대해 강점을 기반으로 기재해 주세요 (500자 이내)}

\bigskip


\underline{\textbf{Notes}}

\begin{itemize}
    \item Key Point
    \begin{itemize}
        \item \#생각한대로 \#원하는대로 \# 개발 잘할수있지?
        \item Java, Python, Javascript, PHP 언어중 한개
        \begin{itemize}
            \item 저는 Javascript을 2년동안 제 전 회사 SiteMax Systems에서 Angular.js와 사용했습니다
            \item 저는 react.js 파이톤 back-end 기반 앱을 만들고 있습니다
        \end{itemize}
        \item 상대방의 그리고 동료의 의견을 잘 이해하고 좋은 방향으로 이끌어 나갈 수 있는 능력
        \begin{itemize}
            \item SiteMax 코드 리뷰때 경청하고 작게 논의하여 기존 앱을 개발하였음
        \end{itemize}
        \item 요구사항에 맞추어 생각한 대로 구현할 수 있는 정도
        \begin{itemize}
            \item 지금 토론토 대학에서도 필요 문구사항을 읽고 요구사항에 맞추어 프로그램을 만들고 있습니다
            \item SiteMax 에 있었을때 300 개가 넘는 mini web-application을 요구사항에 맞추어 만들었습니다
        \end{itemize}
        \item 새로운것에 부담감이 없아야 함
        \begin{itemize}
            \item 저는 새로운것에 배우는것을 즐거워 하는 사람입니다
            \item 토론토 대학원에 가고싶었고 3학년 코스를 들어야 했습니다.
            \item 그중에 하나가 OS 프로그래밍 과목을 들어야 했는데 C 언어 지식이 필요했습니다.
            \item 저는 C 언어를 제 여인과 함께 \texttt{K.N King's C programming} 문제를 풀어 공부를 하였고 그 결과 지금 저는 토론토 대학에서 OS을 공부하고 있습니다
        \end{itemize}
        \item 함께일해요***
        \item 기존에 잘 동작된것들이 동작이 잘 안된다
    \end{itemize}
\end{itemize}




\end{document}